\documentclass[11pt]{article}
\usepackage[utf8]{inputenc}
\usepackage{amsfonts}
\usepackage{natbib}
\usepackage{graphicx}
\usepackage{amsmath}
\usepackage{amssymb}
\usepackage{mathrsfs} % Cursive font
\usepackage{graphicx}
\usepackage{ragged2e}
\usepackage{fancyhdr}
\usepackage{nameref}
\usepackage{wrapfig}
\usepackage{listings}
\usepackage{enumitem}
\usepackage{soul}
\usepackage{xcolor}
\usepackage[colorlinks=true, linkcolor=black, citecolor=blue, urlcolor=blue]{hyperref}
\usepackage{multicol}
\usepackage[spanish]{babel}

% to set spacing between lines
\usepackage{setspace}

% to rotate content 90 degrees
\usepackage{lscape}

% defining command for the curly arrow
\newcommand{\curly}{\mathrel{\leadsto}}

\usepackage{ulem}    % Para tachar texto

\usepackage{mathtools}
\usepackage{xparse} \DeclarePairedDelimiterX{\Iintv}[1]{\llbracket}{\rrbracket}{\iintvargs{#1}}
\NewDocumentCommand{\iintvargs}{>{\SplitArgument{1}{,}}m}
{\iintvargsaux#1}
\NewDocumentCommand{\iintvargsaux}{mm} {#1\mkern1.5mu,\mkern1.5mu#2}

\makeatletter
\newcommand*{\currentname}{\@currentlabelname}
\makeatother

\usepackage[a4paper,hmargin=1in, vmargin=1.4in,footskip=0.25in]{geometry}

\lstset{
    basicstyle=\ttfamily, % Cambia esta línea para modificar la fuente
    % Aquí puedes agregar otras configuraciones de listings
}

%\addtolength{\hoffset}{-1cm}
%\addtolength{\hoffset}{-2.5cm}
%\addtolength{\voffset}{-2.5cm}
\addtolength{\textwidth}{0.2cm}
%\addtolength{\textheight}{2cm}
\setlength{\parskip}{8pt}
\setlength{\parindent}{0.5cm}
\setlength{\headheight}{13.6pt}
\linespread{1.25}

\pagestyle{fancy}
\fancyhf{}
\rhead{Caporalini, Joaquín}
\lhead{Seguridad informática}
\rfoot{\vspace{1cm} \thepage}

\newcommand{\comentario}[1]{}

\renewcommand*\contentsname{\LARGE Índice}

\begin{document}
\thispagestyle{plain}
\begin{spacing}{2.0}
\noindent\textbf{\Huge 
    Practica 1:\\ Introducción a la\\ seguridad informática\\ y la confidencialidad
}
\end{spacing}

\noindent{\huge Entrega de ejercicios seleccionados de la práctica}
\begin{multicols}{2}
    \begin{flushleft}
        De \textit{Joaquín Caporalini}\\ para el curso \textit{Seguiridad informática}\\
        Dictado por \textit{Cristiá, Maximiliano} y \textit{Hernández, Alejandro}
    \end{flushleft}
    \columnbreak
    \begin{flushright}
        \today
    \end{flushright}
\end{multicols}


\section*{2. Uso de e-mails corporativos}
\subsection*{Enunciado}

Las empresas usualmente restringen a sus empleados el uso de e-mails para fines laborales, pero permiten un mínimo uso para razones personales.

\begin{enumerate}
    \item Piense cómo podría una empresa detectar un uso excesivo de e-mails personales, sin tener que leerlos.
    \item Intuitivamente, parece razonable evitar TODO uso personal de e-mail. Explicar porque la mayoría de las empresas no hacen esto.
\end{enumerate}

\subsection*{Propuesta}

Es posible implementar tácticas para intentar mitigar el uso de e-mails corporativos para uso personal sin leer el contenido del mail enviado. Los correros van acompañados de metadata que es utilizable para diseñar filtros. Algunas de las técnicas a implementar:

\begin{itemize}
    \item Comparar el proveedor de mail y el destinatario contra una base de datos de clientes o correos bloqueados, un ejemplo de correo bloqueado podría ser el de la competencia.
    \item Filtrado por hora de envío y la ip que hizo la petición de enviar el e-mail al servidor de correo.
    \item Medir la cantidad de interacciones entre el usuario y los correos de la empresa.
    \item Utilizar sistemas de predicción estadísticos como los utilizados para mitigar los fraudes con tarjetas o el correo SPAM.
    \item Entrenar modelos de IA para clasificar paquetes con toda la metadata disponible para los mails de la empresa.
    \item 
\end{itemize}

Las desventajas de implementar sistemas de este tipo son visibles en algunos de los campos que se nombraron (y por eso no se usan cada vez menos con tarjetas y en el marcado de mails como SPAM). Marcar e-mails como \textit{personales} y evitar su envío o recepción puede conllevar en la \textbf{perdidas} de nuevas ventas o la perdida de información importante y urgente. Además, la generación de falsos positivos suele llevar a una baja de la \textbf{usabilidad} por lo que los empleados terminarían usando otros medios para las comunicaciones y no las direcciones de la empresa. La implementación de estos filtros tiene \textbf{costos} aparejados que también deben ser tenidos en cuenta.

\section*{7. Compromiso de confidencialidad e integridad}
\subsection*{Enunciado}
Dé un ejemplo de una situación en la cual un compromiso de confidencialidad conduce a un compromiso de integridad.
\subsection*{Propuesta}
Supongamos el caso hipotético en el cual un gerente deja sus credenciales de acceso expuestas físicamente o digitalmente, puesto que son \textit{muy complicadas de recordad} se pegó una nota con el \textbf{usuario} y la \textbf{contraseña}(que se asume secreta) en su escritorio. Dentro de las capacidades que tiene el gerente es la de modificar las ordenes de trabajo y el manejo de caja, además de poder ver los perfiles de los clientes y modificarlos.

Si dicho usuario y contraseña son comprometidos, perdiendo su estado de \textbf{confidenciales}, se podría modificar la información sensible al funcionamiento de la organización, dando un fallo de \textbf{integridad}.


\section*{15. Programar en Haskell}
\subsection*{Enunciado}

\textbf{a)} Implementar en Haskell una función:
    
\begin{lstlisting}
grant :: User -> File -> Access -> Boole
grnat user file bole = undefine
\end{lstlisting}

\noindent Que, dado un usuario (sujeto), un archivo (objeto) y un tipo de acceso (lectura, escritura), devuelva un booleano indicando si se puede llevar adelante el acceso, siguiendo la política del modelo Bell-LaPadula.

\textbf{OBS}: \textit{puede agregar parámetro(s) adicional(es) con la información necesaria para poder devolver lo que corresponda.}

\textbf{b)} Si no lo hizo en el inciso anterior, modifique la función de manera que  mantenga el estado de archivos accedidos (abiertos), para tener en cuenta también la propiedad \textbf{*}.

\subsection*{Propuesta}

Para resolver ambos enunciados se propone la siguiente implementación copiando la especificación formal hecha en \textbf{Z} en la teoría.

Para \texttt{leer} nos debemos de fijar:
\begin{enumerate}
    \item Obtenemos los archivos abiertos para el \texttt{user}.
    \item Buscamos los permisos para \texttt{user}. 
    \item Buscamos los permisos para \texttt{file}. 
    \item Hacemos las verificaciones de la especificación:
        \begin{itemize}
            \item El \texttt{user} es del sistema.
            \item El \texttt{file} es del sistema.
            \item no está el El \texttt{file} abierto ya para el \texttt{user} en ese modo.
            \item Se poseen los permisos suficientes
            \item Se verifica la condición \textbf{*}. 
        \end{itemize}
\end{enumerate}

\begin{lstlisting}
grant :: SecState -> User -> File -> Access -> Bool
grant (S sc oobj) user file RM
  = let oobju = findWithDefault [] user oobj
        scu = sc ! user
        scf = sc ! file 
    in user `member` sc && file `member` sc &&
    notElem file [rf | (rf, acc) <- oobju, acc == RM] &&
    scu `dominates` scf &&
    and [(sc ! wf) `dominates` scf | (wf, acc) <- oobju, acc == WM]
\end{lstlisting}

Para \texttt{Escribir} nos debemos de fijar:
\begin{enumerate}
    \item Obtenemos los archivos abiertos para el \texttt{user}.
    \item Buscamos los permisos para \texttt{user}. 
    \item Buscamos los permisos para \texttt{file}. 
    \item Hacemos las verificaciones de la especificación:
        \begin{itemize}
            \item El \texttt{user} es del sistema.
            \item El \texttt{file} es del sistema.
            \item no está el El \texttt{file} abierto ya para el \texttt{user} en ese modo.
            \item Se poseen los permisos suficientes
            \item Se verifica la condición \textbf{*}. 
        \end{itemize}
\end{enumerate}

\noindent Para el caso de las escrituras
\begin{lstlisting}
grant (S sc oobj) user file WM
  = let oobju = findWithDefault [] user oobj
        scu = sc ! user
        scf = sc ! file 
    in user `member` sc && file `member` sc &&
    notElem file [wf | (wf, acc) <- oobju, acc == WM] &&
    scu == scf &&
    and [scf `dominates` (sc ! rf) | (rf, acc) <- oobju, acc == RM]    
\end{lstlisting}

Sindo así que los casos son muy similares.

\section*{21.3. Flujo indebido}
\subsection*{Enunciado}

Determinar si hay flujo indebido de información en cada uno de estos programas.

\begin{lstlisting}
-- Prog 3
x = readH();
writeL("Loading...");
while (x-- > 100)
    writeL(".");
writeH(x);
\end{lstlisting}

\subsection*{Propuesta}

En este ejemplo \textbf{hay} flujo indebido de información. El contenido de la variable \texttt{x} en si mismo no es \textit{filtrado}, pero si su uso, rompiendo la idea de que los usuarios de \textit{bajo} nivel no deben de ser interferidos por las acciones de nivel superior.

\section*{22.3. }
\subsection*{Enunciado}
Para los programas del ejercicio anterior, explique cómo funcionaría la ejecución bajo el sistema de seguridad por multi-ejecución.
\subsection*{Propuesta}

Tomemos la traza del \textbf{programa 3} (el utilizado en el ejercicio anterior)

\begin{enumerate}
    \item El programa seria iniciado con clasificación \textbf{L}.
    \item Se ejecutan sentencias de ese nivel hasta llegar a la primera linea mostrada.
    \item Al intentar ejecutar
\begin{lstlisting}
x = readH();
\end{lstlisting}
    el programa realiza un \texttt{fork()}. En proceso padre continua con permisos \textbf{L} y el nuevo sigue su ejecución con permisos \textbf{H}. 
    \item Según el nivel pararan distintos comportamientos para las sentencias
\begin{lstlisting}
writeL("Loading...");
while (x-- > 100)
    writeL(".");
\end{lstlisting}
    \begin{itemize}
        \item \textbf{L}: El programa imprimirá los mensajes.
        \item \textbf{H}: El programa no podrá imprimir nada porque lo haría en un nivel inferior.
    \end{itemize}
    \item Llega el momento de ejecutar la ultima sentencia 
\begin{lstlisting}
writeH(x);
\end{lstlisting}
    \begin{itemize}
        \item \textbf{L}: este programa no vera ninguna modificación.
        \item \textbf{H}: El programa escribe el valor de la variable puesto que estamos en el entorno \textbf{H} y se hace en un lugar clasificado como \textbf{H}.
    \end{itemize}
\end{enumerate}


\section*{23.4. }
\subsection*{Enunciado}

Resuelva los desafíos propuestos en el siguiente link: \url{http://ifc-challenge.appspot.com/} Para los que pueda resolver, explique las conclusiones a las que arribe, por ejemplo:

\begin{itemize}
    \item Qué tipo de ataque logró llevar adelante?
    \item Qué limitación tiene el sistema de reglas?
    \item Qué cambio habría que hacerle para impedir el ataque?
\end{itemize}

\subsection*{Propuesta}

El ejercicio que elegí fue el 6 donde el ataque venia de la mano de \texttt{try <stm> catch <stm> } y \texttt{throw;}.

Pude reproducir el ataque de la siguiente forma:

\begin{lstlisting}
l = true;
try{
  if (h) throw;
  l = false;
} catch
  skip;
\end{lstlisting}

En este ataque a la confidencialidad una variable de nivel de seguridades superior \texttt{h} es vulnerada puesto qeu su valor es copiado en una variable con nivel de seguridad inferior \texttt{l}. De esta forma podríamos filtrar secretos para desclasificarlos. Una forma de evitar este ataque es que si se accede a la lectura de variables seguras no se permita escribir en esos entornos otras variables con niveles inferiores, elevando así a \textit{high} los permisos del \textbf{try} y \textbf{catch}

\section*{A. Apendice de código}

A continuación el código completo para el ejercio 15.

\begin{lstlisting}
import Data.Map

type Cat = String
type Obj = Int
type SL = Int
type SC = (SL, [Cat])

type User = Obj
type File = Obj

dominates :: SC -> SC -> Bool
dominates (sl1, scat1) (sl2, scat2) = sl2 <= sl1 && scat2 `subset` scat1
  where
    subset list1 list2 = all (`elem` list2) list1

data Access = RM | WM deriving (Show, Eq)

data SecState = S {
  sc :: Map Obj SC,
  oobj :: Map User [(File, Access)]
}

grant :: SecState -> User -> File -> Access -> Bool

grant (S sc oobj) user file RM
  = let oobju = findWithDefault [] user oobj in
        scu = sc ! user
        scf = sc ! file 
    in user `member` sc && file `member` sc &&
    notElem file [rf | (rf, acc) <- oobju, acc == RM] &&
    scu `dominates` scf &&
    and [(sc ! wf) `dominates` scf | (wf, acc) <- oobju, acc == WM]

grant (S sc oobj) user file WM
  = let oobju = findWithDefault [] user oobj in
        scu = sc ! user
        scf = sc ! file 
    in user `member` sc && file `member` sc &&
    notElem file [wf | (wf, acc) <- oobju, acc == WM] &&
    scu == scf &&
    and [scf `dominates` (sc ! rf) | (rf, acc) <- oobju, acc == RM]
\end{lstlisting}

%%\section*{}
%%\subsection*{Enunciado}
%%\subsection*{Propuesta}

% \bibliographystyle{alpha}
% \bibliography{referencias}

\end{document}
