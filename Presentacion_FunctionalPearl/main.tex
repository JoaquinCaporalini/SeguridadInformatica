\documentclass{beamer}
\usepackage{lmodern}
\usepackage{anyfontsize}
\renewcommand{\normalsize}{\fontsize{14}{16}\selectfont}

\usepackage{tikz}

\usepackage{xcolor} % Cargar el paquete para colores

\usepackage{listings}
\lstset{
    language=Haskell,
    basicstyle=\ttfamily,
    % keywordstyle=\color{blue},
    commentstyle=\color{green!40!black},
    stringstyle=\color{orange},
    showstringspaces=false,
    breaklines=true,
    keywordstyle = [3]{\color{blue}},
    morekeywords = [3]{L},
    keywordstyle = [4]{\color{red}},
    morekeywords = [4]{H}
}

\usetheme{Madrid}
\usecolortheme{default}


%Information to be included in the title page:
\title[Functional Pearl] %optional
{Functional Pearl: \\ Two Can Keep a Secret, If One of Them Uses Haskell}

\subtitle{
}

\author[Joaquin Caporalini] % (optional, for multiple authors)
{
    Alejandro Russo
}

\institute[LCC - FCEIA] % (optional)
{
    Facultad de Ciencias Exactas, Ingeniería y Agrimensura\\Universidad Nacional de Rosario
}

\date[] % (optional)
{\textbf{} \\ Joaquín Caporalini\\Febrero 2025}

\begin{document}

\frame{
    \titlepage
}

% \begin{frame}{Ejemplo con Rectángulos}
%     \begin{tikzpicture}[remember picture, overlay]
%         % Rectángulo con borde rojo y sin relleno en la parte superior izquierda
%         \draw[red, line width=2pt] (0,3) rectangle (3,4);
        
%         % Rectángulo con borde rojo y sin relleno en la parte inferior derecha
%         \draw[red, line width=2pt] (7,-2) rectangle (10,-1);
        
%         % Rectángulo con borde rojo y sin relleno en el centro
%         \draw[red, line width=2pt] (4,1) rectangle (6,2);
%     \end{tikzpicture}
    
%     \centering
%     \textbf{Ejemplo de rectángulos con borde rojo en Beamer}
% \end{frame}

\begin{frame}
    \frametitle{Un encargo para Alice...}
    Construir un gestor de contraseñas simples.\pause Con la función agregada de contraseñas comunes.
    \begin{center}
        \includegraphics[scale=0.65]{codigo_alice.png}
    \end{center}\pause
    \begin{tikzpicture}[remember picture, overlay]
        % Rectángulo con borde rojo y sin relleno en el centro
        \draw[red, line width=1pt] (3.85,2) rectangle (6.6,1.75);
    \end{tikzpicture}
    El código de Bob:
    \begin{center}
        \includegraphics[scale=0.65]{codigo_bob.png}
    \end{center}
\end{frame}

\begin{frame}
    \frametitle{Un encargo para Alice...}
    \textbf{¿Qué debería hacer Alice?}\newline

    Para proteger recursos no alcanza con listas negras (o blancas), sino de asegurar que la {\textbf<2->{información fluye}} solo hacia los lugares adecuados.\newline
    \pause

    \begin{flushright}
        \it{¿Cómo se logra eso?}
    \end{flushright}

\end{frame}

\begin{frame}
    \frametitle{Mandatory Access Control e Information-Flow Control}
    Lo logra aplicando no interferencia\newline

    Provienen de la investigación
    \begin{itemize}
        \item MAC: sistemas operativos
        \item IFC: lenguajes de programación
    \end{itemize}
    \hspace{1cm}

    La propuesta es aprovechar \textbf{conceptos de lenguajes de programación} para implementar mecanismos similares a MAC mediante la creación de una \textbf{API monádica} que protege \textbf{confidencialidad estáticamente}.

\end{frame}

\begin{frame}
    \frametitle{Látice de seguridad}
    \textbf{¿Cómo se etiquetan los datos?}
    Están organizadas en un látice de seguridad.
    \begin{center}
        \includegraphics[scale=0.8]{figure1.png}
    \end{center}
    La información no pueda ir de entidades secretas a públicas (no interferencia): $\textcolor{blue}{L} \sqsubseteq \textcolor{red}{H}$ y $\textcolor{red}{H} \not\sqsubseteq \textcolor{blue}{L}$.
\end{frame}

\begin{frame}
    \frametitle{Familia de mónadas MAC}
    Encapsula acciones de IO para que no revelen información\newline

    Está indexada por una etiqueta de seguridad indicando la sensibilidad de sus resultados monádicos.

    \begin{center}
        \includegraphics[scale=0.7]{figure2.png}
    \end{center}
\end{frame}

\begin{frame}{Recursos etiquetados}
    
    \begin{center}
        \includegraphics[scale=0.7]{figure3.png}
    \end{center}
    \begin{columns}
        \column{0.5\textwidth}
            \begin{center}
                \includegraphics[scale=0.7]{figure4.png}
            \end{center}
        \column{0.5\textwidth}
        Recursos etiquetados tienen en cuneta:
        \begin{itemize}
            \item Carácter de los datos
            \item Origen
            \item Destino
        \end{itemize}
        \pause Permite llevar a problema lectura/escritura
    \end{columns}
\end{frame}

\begin{frame}
    \frametitle{Lift de las acciones de IO (Mantener un secreto)}
    Siguiendo los principios de \textit{no read-up} y \textit{no write-down} se extiende la TCB\footnote{Trusted Computing Base} con funciones que \textbf{elevan}\footnote{Función equivalente que trabaja en los términos de la mónada} las acciones \textit{IO}.

    \begin{center}
        \includegraphics[scale=0.7]{figure5.png}
    \end{center}
\end{frame}

\begin{frame}
    \frametitle{Expresiones etiquetadas}

    Posible etiquetado.

    \begin{center}
        \includegraphics[scale=0.8]{figure6.png}
    \end{center}

    Notar el sinónimo de tipo como abreviatura  
\end{frame}

\begin{frame}[fragile]
    \frametitle{Uniendo miembros de la familia}
    Si Bob usase \textit{MAC} su función podría tener el tipo

    \begin{lstlisting}
common_pwds :: Labeled H String -> 
               MAC L (MAC H Bool)
    \end{lstlisting}
    
    En este caso la anidación de computaciones es manejable, pero habrá casos para los que tal vez no, por eso se introduce:

    \begin{center}
        \includegraphics[scale=0.8]{figure7.png}
    \end{center}
\end{frame}

\begin{frame}{Añadiendo referencias (Mutavilidad)}
    \begin{center}
        \includegraphics[scale=0.7]{figure8.png}
    \end{center}

    Las funciones se elevan a la mónada $MAC \ l$ envolviéndolas con $new^{TCB}$, $read^{TCB}$ y $write^{TCB}$ respectivamente. \pause

    Estos pasos se generalizan para obtener interfaces seguras de diversos tipos, como veremos más adelante.
\end{frame}

\begin{frame}{Manejo de errores (Excepciones)}
    \begin{center}
        \includegraphics[scale=0.7]{figure9.png}
    \end{center}

    %Las excepciones se capturan en el mismo \textbf{tipo} de miembro de la familia donde fueron arrojadas.\newline\pause

    \textbf{Pero, ¿qué pasa con las construcciones con $join^{MAC}$?}
    
    Pueden comprometer la seguridad...
    
    Una acción \textcolor{red}{H} lanzar excepciones y evitar acciones de nivel bajo con la función $join^{MAC}$.
\end{frame}

\begin{frame}{Como lo explotaría un atacante (Bob)}

    \begin{center}
        \includegraphics[scale=0.7]{codigo_bob2.png}
    \end{center}

    \begin{center}
        \includegraphics[scale=0.7]{codigo_bob3.png}
    \end{center}
\end{frame}


\begin{frame}{Nuevo $join^{MAC}$}
    Se redefine $join^{MAC}$ de manera tal que la propagación de excepciones entre miembros de la familia quede deshabilitada.

    \begin{center}
        \includegraphics[scale=0.7]{figure10.png}
    \end{center}
\end{frame}

\subsection{El elefante (encubierto) en la habitación}
\begin{frame}{El elefante (encubierto) en la habitación}
    Existe un canal encubierto: la \textbf{no terminación}...\newline

    En un entorno secuencial, la manera más efectiva de explotar un canal encubierto de no-terminación es a través de \textbf{fuerza bruta}, por lo que no hay gran ancho de banda si el universo donde buscar es lo \textbf{suficientemente grande}. \newline
    
    En ese caso se puede omitir el análisis de estos canales encubiertos. (\textit{problema de la parada})

    \vspace{0.5cm}
    \pause
    \begin{flushright}
        \it{¿Pero qué sucede cuando hay concurrencia?}
    \end{flushright}
    
\end{frame}

\begin{frame}{\textit{fork} como primitiva (Concurrencia)}
    Alice añade concurrencia extendiendo la API así:

    \begin{center}
        \includegraphics[scale=0.8]{codigo_alice2.png}
    \end{center}

    \textbf{¿Qué ataque puede intentar Bob?}\newline

    Explotar el canal encubierto de la no terminación de programas.
\end{frame}

\begin{frame}{Bob con concurrencia}
    \begin{center}
        \includegraphics[scale=0.7]{codigo_bob4.png}
    \end{center}

    \begin{center}
        \includegraphics[scale=0.7]{codigo_bob5.png}
    \end{center}

    \texttt{loop} es una función que no termina
\end{frame}

\begin{frame}{Solución}
    El problema viene de la interacción de $join^{MAC}$ con $fork^{MAC}$.

    Pero, ¡se puede reemplazar a $join^{MAC}$ por $fork^{MAC}$!

    \begin{center}
        \includegraphics[scale=0.8]{figure11.png}
    \end{center}

    Aunque se haya removido\footnote{Notar que queda un parecido con \textit{multi-ejecución segura}} $join^{MAC}$ se pueden combinar computaciones con las referencias seguras introducidas previamente.
\end{frame}

% \begin{frame}{MVars (primitivas de sincronización)}
%     Se extiende \textbf{MAC} con \textit{MVars} ---una abstracción de sincronización muy utilizada en Haskell--- similar a como se hizo con referencias.
    
%     \begin{center}
%         \includegraphics[scale=0.7]{figure12.png}
%     \end{center}

%     \textbf{TODAS} las acciones provocan el efecto secundario de lecto/escritura.
% \end{frame}

\begin{frame}{Comentarios finales}
    \begin{itemize}
        \item<1-> Las abstracciones que provee Haskell y, en general, los lenguajes funcionales, son muy buenas para enfrentarse a los desafíos de seguridad actuales.

        \item<2-> La corrección de \textbf{MAC} depende de la seguridad de tipos y la encapsulación de módulos de Haskell. MAC utiliza Safe Haskell al compilar código no confiable.
        
        \item<3-> La descalcificación intencional no es tratada en este paper sin embargo existen varios enfoques.
        
        \item<4-> Para el ejemplo se uso un etiquetado de dos niveles. Se encontraron formas de usar el sistema de tipos cerrado de GHC para generar extensiones.
    \end{itemize}
\end{frame}

\end{document}